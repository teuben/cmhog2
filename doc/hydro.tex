%\documentstyle[11pt,epsf]{article}
%\documentstyle[11pt,epsf]{report}
%\documentclass[11pt,epsf]{report}
\documentclass[11pt,epsf]{article}

\pagestyle{myheadings}

% Some definitions I use in these instructions.

\def\emphasize#1{{\sl#1\/}}
\def\arg#1{{\it#1\/}}
\let\prog=\arg

\def\edcomment#1{\iffalse\marginpar{\raggedright\sl#1\/}\else\relax\fi}
\marginparwidth 1.25in
\marginparsep .125in
\marginparpush .25in
\reversemarginpar

\title{Using CMHOG with NEMO and MIRIAD}
\author{Peter Teuben}
% \affil{Astronomy Department, University of Maryland, College Park, MD 20742, USA}


\begin{document}

\maketitle


\begin{abstract}

An overview is given how to use and extend 
the Piner, Stone \& Teuben (1995, ApJ 449, 508) hydro code
{\tt cmhog}, and in particular how HDF output data can be used with
NEMO and MIRIAD for some basic visualization and data analysis.
Only single runs are covered, restart runs are not discussed.

\end{abstract}

\section{Introduction}

The Piner, Stone and Teuben ``{\tt cmhog}'' is a PPM code, 
that was adopted to run a 2-dimensional polar-grid gas flow in a 
barred galaxy. Because of symmetry, only half of the plane is
simulated, proper boundary condition will take care of the
other half. The bar is situated along the Y-axis, with 
counter clockwise gas flow. Initially the bar is absent, but
slowly turned on  while keeping the total mass constant.

\section{The ``cmhogin'' input file}

The {\tt cmhog} program uses a classic, but not often used, method of parameter input
that is a little more rubust to errors than the more often (ab)used method
of reading information from standard input. It's called {\tt namelist}\index{namelist},
and works roughly like magically reading a FORTRAN named common block. 

\subsection{For the programmer}

For the programmer it is very convenient to pass parameters via a namelist.
After opening the unit somewhere at the beginning of the program
(cmhog does this in mstart.src) various subroutines can simply read
the namelist  that they need.

\begin{verbatim}
    open(unit=1,file='cmhogin' ,status='old')                     ::mstart.src

    real*8 amp,amode,n,aob,qm,rhoc				  ::galaxy.src
    real   bartime
    namelist /pgen/ amp,npar,amode,n,aob,rl,qm,rhoc,bartime       
    read (1,pgen)                                                 

    logical lgrid
    namelist /ggen2/ nbl,ymin,ymax,igrid,yrat,dymin,vgy,lgrid	  ::ggen.src
    namelist /ggen3/ nbl,zmin,zmax,igrid,zrat,dzmin,vgz,lgrid
    read (1,ggen2)




\end{verbatim}

\subsection{For the user}

For the user this means all parameters are set in a simple ascii file
named ``{\tt cmhogin}'' that can be edited. This file must be in the
current directory when {\tt cmhog} starts to run (see also
{\tt runcmhog} below).

\begin{verbatim}

 $rescon  $
 $hycon idiff=1,ifltn=1,tlim=2.0 $
 $ggen2 nbl=132,ymin=0.1,ymax=16.0,yrat=1.03926991,igrid=1,lgrid=.true. $
 $ggen3 nbl=80,zmin=-1.5708,zmax=1.5708,igrid=1,lgrid=.true. $
 $ijb $
 $ojb $
 $ikb $
 $okb $
 $eos ciso=5.0 $
 $pgen amp=10.0,amode=1.0,n=1.0,aob=2.5,rl=6.0,qm=4.5e4,rhoc=2.4e4,bartime=0.1 $
 $spiral spamp=0.0,spang=-0.5236,spsc=1.0,sppat=32.74,sr=4.0,pc=2.328 $
 $iocon dthdf=0.1,dtmovie=5.0,dthist=0.01 $
 $mlims cma1=3.0,cmi1=-2.0,cma2=200.0,cmi2=-200.0,cma3=200.0,cmi3=-200.0 $

\end{verbatim}

When {\tt cmhog} finishes, the file {\tt cmhogout} will reflect the status
of the namelist, but you cannot re-use it (Jim, why is that??? why is the
format of cmhogin and cmhogout different???)


The most common ones to modify are (geometry in {\tt ggen2} and {\tt ggen3}
will be dicussed later)

\begin{itemize}

\item
{\tt pgen::aob}
Axis ratio $a/b$ of the bar, will be > 1.

\item
{\tt pgen::rl}
Lagrangian radius, in kpc. This sets the pattern speed.

\item
{\tt pgen::qm}
Quadropole moment, this sets the mass fraction of the bar.

\item
{\tt pgen::rhoc}
Central density, this sets 

\item
{\tt pgen::bartime}
Time (in Gyr)

\item
{\tt iocon::dthdf}
Timestep for HDF data dumps (in Gyr)

\item
{\tt icon::dtmovie }
Timestep for {\tt mde/mvt/mvr} data dumps (in Gyr)

\item
{\tt icon::dthist}
Timestep in {\tt history} file (in Gyr)

\item
{\tt hycon::tlim=2.0}
Stop time of the integration (in Gyr)

\end{itemize}


\subsection{Choice of the grid}

The grid is a polar grid, and set by the {\tt ggen2} (radial) and 
{\tt ggen3} (angular) namelist entries, of which the {\tt min} and
{\tt max} variables control the centers of the cells:

\begin{verbatim}
 $ggen2 nbl=132,ymin=0.1,ymax=16.0,yrat=1.03926991,igrid=1,lgrid=.true. $
 $ggen3 nbl=80,zmin=-1.5708,zmax=1.5708,igrid=1,lgrid=.true. $
\end{verbatim}

although there are 7 essential variables in these 
two namelist entries, not all should be choosen freely. 
One possibilty is to choose the 3 radial extent variables
and derive the remainder from some constraints :


\begin{enumerate}
\item
choose  {\tt ggen2::ymin} = $y_{min}$, the inner boundary of the grid (in kpc)
??? is this the radial center of the inner most cell, or the inner boundary???
if it's the center, what is really the correct formulae for the stuff derived below.
that will have to change slightly.

\item
choose {\tt ggen2::ymax} = $y_{max}$, the outer boundary of the grid (in kpc)
??? is this the radial center of the out most cell, or the outer boundary???

\item
choose {\tt ggen2::nbl} = $N_y$, the number of radial zones.

\item
compute {\tt ggen2::yrat} = $y_{rat}$ from the following formulae:
$$
	 y_{rat} = {y_{max} \over y_{min}}^{1/{N_y}}
$$

\item
Compute {\tt ggen3::nbl} = $N_z$, the number of angular zones, such that cells are 
close to being square from the following relationship that equates the two:
$$
	{{\pi y_{min}} \over N_z} = (y_{rat} - 1 ) y_{min}
$$
or
$$
N_z =   {  {\pi y_{min}  }   \over  { (y_{rat} - 1 ) y_{min} } }
$$

\item
Compute {\tt ggen3::zmax} = $z_{max}$ from
$$
	( \pi - z_{max} ) 2 = {\pi \over N_z}, ~~~~~ or,~~~~~~~
	z_{max} = \pi ( 1 - {1 \over {2 N_z}})
$$

\item
Compute {\tt ggen3::zmin} = $z_{min}$ from
$$
	( z_{min} + \pi ) 2 = {\pi \over N_z}, ~~~~~~ or,~~~~~~~
	z_{min} = - z_{max} 
$$



\end{enumerate}

In case we got the boundaries wrong, the following is more appropriate: 
$r_0$ is the inner boundary, $r_N$ the outer, and $f=1+\epsilon$ the
geometric factor called {\tt yrat} earlier.
$$
r_i = r_{i-1} f, ~~~ i=1..N
$$
thus
$$
r_N = r_0 f^N
$$
and since
$$
r_{min} = (r_0+r_1)/2,     ~~~~~ r_{max} = (r_{N-1}+r_N)/2
$$
we get
$$
r_{min} = r_0 (1+f)/2,   ~~~~~ r_{max} = r_{min} f^{N-1}
$$
or
$$
r_0 = { 2 \over  { 1+f}}  r_{min}, ~~~~  r_N = r_O f^N
$$


Here are some formulae that we know are correct:

$$
      r_1 = r_0 + dr, ~~  r_2 = r_1 + dr f  ~~...~~ r_N = r_{N-1} + dr f^{N-1}
$$
or following standard summation series (see e.g. G\&R, or Abramowitz)
$$
	r_N = r_0 + dr \sum_{k=0}^{N-1}{f^k} = r_0 + dr { {f^N-1} \over { f - 1 }}
$$
and
$$
     dr =  (r_N - r_0) { {f-1} \over { f^N - 1 }}
$$



\section{Output}

For the remainder of the discussion, we will assume all data produced
by {\tt cmhog} is dumped in a run-directory. 
Depending on options given to the program, you will find the following files:

\begin{itemize}

\item
{\tt cmhogin}
Input namelist, that controls the program.

\item
{\tt cmhogout}
Output namelist, not used for anything?

\item
{\tt history}
Ascii table with time in the first column, other columns 
are used for things such as mass loss across the inner and 
outer boundaries etc.etc.

\item
{\tt hdfXXXbg}
Full HDF dataset, in SDS format (Scientific Data Set), typically
with three named ``fields'': the 
{\tt R-VELOCITY}, {\tt PHI-VELOCITY}, and {\tt DENSITY-VELOCITY}.
The NEMO program {\tt tsd} will display the contents of
such SDS files. Each file contains information for one timestep.
{\tt hdf000bf} is normally the first one, for $t=0$ at the beginning
of the simulation.

\item
{\tt mdeXXXbg}
Sample X-Y gridded density in a 8bit deep image

\item
{\tt mvrXXXbg}
Sample X-Y gridded radial velocity in a 8bit deep image

\item
{\tt mdvtXXXbg}
Sample X-Y gridded tangential in a 8bit deep image

\end{itemize}

\section{NEMO programs}

\subsection{runcmhog}

{\tt cmhog} is a program that needs to be run in a clean directory, and you
cannot run another {\tt cmhog} in that directory. That is because the names
of files that {\tt cmhog} produces are FIXED by the program, and cannot be
changed even by the {\tt cmhogin} file (it is also not practical to do that).
To help running {\tt cmhog}, a small pre-processor was written, called
{\tt runcmhog}. It is a little C program available in NEMO with which you
can use a template {\tt cmhogin} file and override parameters and set
a run directory, e.g.

\begin{verbatim}
    % runcmhog -n cmhogin.small run001  aob=2.0
    % runcmhog -n cmhogin.small run002  aob=2.5
    % runcmhog -n cmhogin.small run003  aob=3.0
    % runcmhog -n cmhogin.small run004  aob=3.5
\end{verbatim}

Jianjun Xu (a student of Jim Stone) once write a nice graphical frontend,
based on Tcl/Tk, to parse and generate cmhogin files. The code still
exists, but has not been excersized in a long time.

\subsection{tsd}

\footnotesize\begin{verbatim}
% tsd hdf001bg 
Found 3 scientific data sets in run001/hdf001bg
1: R-VELOCITY AT TIME=1.00E-01(20,37) km/sec  -> [740 elements of type: 5 (FLOAT32)]
2: PHI-VELOCITY AT TIME=1.00E-01(20,37) km/sec  -> [740 elements of type: 5 (FLOAT32)]
3: DENSITY AT TIME=1.00E-01(20,37) Msolar/pc**2  -> [740 elements of type: 5 (FLOAT32)]
\end{verbatim}\normalsize

\subsection{sdsfits}

The individual ``fields'' from an {\tt hdfXXXbg} files can be converted to FITS files,
for visual inspection. Just remember that the first axis is the radial axis, the
second one the angular.  A ring will thus show up as a vertical structure.

\footnotesize\begin{verbatim}
% sdsfits hdf001bg map001vr.fits 1
% sdsfits hdf001bg map001vt.fits 2
% sdsfits hdf001bg map001de.fits 3
\end{verbatim}\normalsize

\subsection{hdfgrid}

{\tt hdfgrid}
\footnote{A good manual page should be available with the man command}
regrids selected gas properties from our 
HDF files into a cartesian coordinate system. 
It is specific to this polar-coordinate problem.

\footnotesize\begin{verbatim}
% hdfgrid hdf001bg map000de.ccd zvar=den
% hdfgrid hdf001bg map000vr.ccd zvar=vr
% hdfgrid hdf001bg map000vt.ccd zvar=vt
% hdfgrid hdf001bg map000vx.ccd zvar=vx
% hdfgrid hdf001bg map000vy.ccd zvar=vy
\end{verbatim}\normalsize

Note that the {\tt vx} and {\tt vy} velocity fields
are computed on the fly. Nearest neighbor cells are
used for bi-linear interpolation. Some assumptions
about the symmetry properties of the variables
have been made.

\section{flowcode}

Although gas flow in a barred galaxy never reaches an
exact steady state, it does approach something
that can perhaps be called a QSSS. However, another approach
to understand the gas flow is take a particular snapshot, and
integrate test particles in the velocity field, i.e.
solve the equations
$$
	{dx \over dt } = V_x(x,y)
$$
and
$$
	{dy \over dt } = V_y(x,y)
$$
NEMO's {\tt flowcode} program does that. For this a special
``vrt'' file (NEMO image format) is needed as input for the
integrator, consisting
of two 2D images with VR and VT, followed by two 1D images
with the coordinates in Radius and Polar Angle (called PHI)
A ``vrt'' map is created with a script {\tt mkvrt}.

\section{The Program}

Files are .src (fortran source to be pre-processed by cpp) and 
.inc (for the preprocessor) and .h (standard fortran includes).

\begin{verbatim}
    main    
        mstart
	    setup
	    grid_generator    (ggen)
	    problem_generator (galaxy)
	    nudt
        dataio
	
	solver
	intchk
	dataio
	special
	nudt
	
	dataio
\end{verbatim}

\end{document}

\footnotesize\begin{verbatim}
\end{verbatim}\normalsize


\begin{verbatim}
\end{verbatim}
